\documentclass[thesis.tex]{subfile}

\chapter{Introduction}
% Limit myself to 2 paras here?

In many engineering domains, finite-difference methods are used to solve differential equations, where an exact solution may be computationally infeasible.
These approximations obviate the need for equations to be solved by hand, saving considerable labour.
Optimising the computations is therefore a natural extension of this task.

% TODO: look up Devito's raison d'etre
For scientists using differential equations in novel ways, tools which can compute partial differential equations directly are helpful, as it enables them to refine their models quickly.
Such tools are less crucial to those employing fluid-dynamical models, which are relatively well-understood and optimised. % TODO: rewrite, cite


\section{Motivation}
% Talk about the big idea in background
% Here: why/who optimise?

It is inefficient and arguably undesirable for scientists principally concerned with modelling to need to understand the structures which speed up their computation. % "understand": think about, etc
As a means of abstraction, compilers are used to transform differential equations into stencils, then code to evaluate the equations.
This reduces the effort required to comprehend the calculations, aiding maintenance and even reproducibility.

Decoupling the understanding of optimisations means that scientists not investigating software performance and compiler technology need not keep up with computer architecture.
New instructions may be introduced (such as vectorisation); not for nothing are compilers able to generate code tailored to the architecture for which a program is compiled.

% TODO
In order to discuss this, we realise that loops are where most computations spend their time...


\section{Devito}
% Optimise how?

% Vertical integration: to simplify this workflow, Devito starts from PDEs
% Devito because it allows scientists to change their models easily
% [Remark: fluid dynamics people never change PDEs, already handcoded & fast]
% Context: other stencil DSLs do time tiling starting from stencils
% Allows us to generate meaningful problem instances (don't write this here)

% TODO: informal
% TODO: Check definition of stencil
In modern usage, such computations may be transformed into \emph{stencils}, which are \textbf{[I don't know how to explain this]}.
There are many compilers optimising stencil kernels with \emph{time-tiling}, an optimisation which we will discuss in detail.
Devito, however, is able to generate computations from the differential equations themselves, achieving a notion of `vertical integration' within the modelling ecosystem.
This is extremely helpful if one is experimenting with models, or continually modelling new problems, as one can change the equations used and rapidly generate the relevant code for their computation.

% TODO: consider mentioning more about the optimisation, polyhedral model etc


\section{Objectives and (intended) contributions}
Devito was able to perform tiling on all dimensions other than time.
The principal objective of this project was to implement tiling over the time dimension, which necessitates an additional transformation to ensure its legality.
% TODO: why it hasn't been solved for Devito

It has been previously demonstrated that extending tiling to the time dimension has the potential to reduce the run time of `some Devito stencil loops by up to 27.5\%' \cite{dylan}.
This project focuses on the implementation of this optimisation, and evaluation of the claim.

In summary, our contributions are:

% TODO: We want a lot more detail in this list
% Add background & related work?
\begin{itemize}
	\item Implementation of skewing/tiling and relevant tests in Devito (hopefully accepted; Chapter XX)
	\item Evaluation of performance (Chapter XX)
\end{itemize}
