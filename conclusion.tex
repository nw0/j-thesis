\documentclass[thesis.tex]{subfile}

\chapter{Conclusion}
\label{ch:conclusion}


\section{Context and review}

This project drew together aspects of compiler transformations, differential equations and computation, computer architecture, and software engineering.
The different perspectives of the fields provided immense insight into the complexity of numerically approximating partial differential equations---a far cry from anything Taylor may have imagined in the 18th century!
These insights culminated in a transformation yielding significant performance gains in Devito's target domain: seismic imaging.
Though time-tiling is a well-established optimisation, this implementation is a necessary and important step for Devito.

\wip{Rewrite; move?}

During our evaluation in Chapter~\ref{ch:evaluation}, we used this abstraction to effortlessly generate entire families of stencils by varying the space order; producing these stencils from the equations is decidedly non-trivial, and would have otherwise been extremely time-consuming.
As a consequence, this project includes the broadest survey of stencils under time-tiling to date.

While the concept of time-tiling is superficially straightforward, its implementation was decidedly not so: unlike interfacing with a `back-end' polyhedral compiler, such as CLooG or Pochoir (Section~\ref{sec:pochoir}) with all its attendant integration problems, implementation within Devito meant understanding the numerous stages of internal representation used to transform a differential equation into code.
Without these layers, Devito would not be able to expose APIs at different levels for more intricate applications; making a transformation that would work with these was a major hurdle.

The performance evaluation was enlightening and challenging in its own ways.
It is easy to be convinced of the benefits of time-tiling from the literature, and another experience altogether to witness real decreases in runtime while watching a wall clock;
mustering the scepticism required to confirm its validity required much deeper reference to computer architecture and computation than implementing the transformation had needed.


\section{Contributions}
A summary of the contributions of this project.

\wip{Results change, proofread this late}
\begin{itemize}
	\item The headline contribution of the project; documentation and implementation of the time-tiling transformation in Devito, with accompanying test cases and auto-tuner enhancements.
	This is a necessary and significant step toward implementing time-tiling for Devito in its full generality, including tiling of imperfect loop nests.

	\item We outlined the remaining work in implementation, including analyses of their importance and difficulty, and their consistency with our implemented transformation.

	\item We presented and critiqued our testing methodology and models, which we used for the performance evaluation of our implemented time-tiling transformation.
	In particular, we devoted attention to numerical verification, the set-up of the test environment, and the modelling of performance.

	\item We demonstrated a runtime decrease of 25\% with the transformation compared to Devito's existing optimisations under real-world applications, particularly the stencils of varying space orders arising from the acoustic wave equation, of great importance to Devito's target domain, seismic imaging.
	In so doing, we validated previous work which demonstrated a runtime reduction of up to 27.5\%~\cite{dylan}.

	\item Further, we developed a model estimating runtime reductions based on the space order of a stencil generated from the family of acoustic wave equation operators.

	\item We proposed an estimator for \emph{arithmetic intensity under time-tiling} that is consistent with Devito's reported arithmetic intensity under spatial tiling.
	Just as importantly, we established bounds for both and showed that the estimators are plausibly close to the true arithmetic intensity of a stencil computation, and that they are consistent with the widely-cited roofline model.

	\item We developed a hypothesis on how the skewing factor affects the runtime of a stencil, which holds in our test architecture, and in particular our experimentation.
\end{itemize}

\section{Future work}
\label{sec:future-work}

Implementation-related: retain in chapter: the problems and why not fixed, move: why/how to fix them -- a lot of detail is possible.

Non-implementation-related: further evaluation, particularly the skewing factor
