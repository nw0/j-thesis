\documentclass[thesis.tex]{subfile}

\chapter{Introduction}
% Dylan has a long exposition of compilers; which I could emulate (but preferably in a more domain-specific way)

In many engineering domains, finite-difference methods are used to solve differential equations, where an exact solution may be computationally infeasible.
These approximations obviate the need for equations to be solved by hand, saving considerable labour.
Optimising the computations is therefore a natural extension of this task.

In modern usage, such computations may be specified as \emph{stencils}, and transformed and optimised by a compiler.
An optimising compiler is highly desirable in such domains, as it removes the need for scientists performing the calculations to understand the structures to speed up their computations.
This reduces the effort required to comprehend the calculations, aiding maintenance and even reproducibility.
Such abstraction is particularly desirable, as computer architecture changes it is unreasonable to expect scientists working in seismic imaging, for instance, to investigate new optimisations, a task better left to those investigating software performance.
New instructions may be introduced (such as vectorisation); not for nothing are compilers able to generate code tailored to the architecture for which a program is compiled.

\section{Loop blocking}
% Remark: don't talk about what loop blocking is, but say why we care about it
% TODO: assume reader knows what a loop nest is?
\emph{Loop blocking} or \emph{tiling} is a well-established technique to exploit data locality in \emph{loop nests}, in which the bulk of such computations reside. % TODO: cite
Blocking is a technique that may be applied regardless of the specific iteration space, as long as no data dependencies cross the boundaries between blocks.
It is also used to enable other optimisations, such as loop-invariant code motion, which may not be appropriate on code which has not been blocked, when the extents of the iteration are too large.

The key objective of this project was to implement blocking over the time dimension in Devito, which necessitated an additional transformation to ensure its legality.
Devito was able to perform blocking on all dimensions apart from time.
% TODO: why it hasn't been solved for Devito

It has been previously demonstrated that extending blocking to the time dimension has the potential to reduce the run time of `some Devito stencil loops by up to 27.5\%' \cite{dylan}.
This project focuses on the implementation of this optimisation, and evaluation of the claim.

\section{(Intended) Contributions}
\begin{itemize}
	\item Implementation of skewing/tiling and relevant tests in Devito (hopefully accepted)
	\item Evaluation of performance
\end{itemize}
