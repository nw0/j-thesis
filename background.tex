\documentclass[thesis.tex]{subfile}

% Purpose of background chapter
% 1. Background to understand thesis (short)
% 2. "Related work", or intro++: research context, where the boundaries are

% Examiners will know about: PDEs, maybe handcoding

\chapter{Background}

% TODO: \ref sections here
We first provide an overview of loop nest optimisations, techniques, and analyses that we have applied, with specific reference to the polyhedral model. % TODO: cite
This forms the basis and context for the entire report, and in particular the survey of related work, which composes the remainder of this chapter.

% TODO: grandiose
This project extends a well-established idea from compiler theory, \emph{tiling}, to another dimension (time) in Devito.
This has traditionally been a challenging problem, as evaluating data dependences efficiently is beset with difficulties.

% We have a tradeoff between computation and memory usage
% Big IDEA: can use more memory, less computation ("this really should be true")
% Project: optimise memory usage
% Why exciting: one step to proving the idea


\section{Loop tiling}
% TODO: mention that iteration spaces under consideration are (nice rectangular continuous) matrices

\subsubsection{Optimisations on loop nests}
The bulk of computation for finite difference methods lies in loops. % TODO: cite
Loop nest optimisations seek to transform a loop, possibly changing its execution order to use data locality, parallelism, or otherwise avoid unnecessary operations.

% TODO
\paragraph{Data locality}
Something about caches: yet another thing I don't know how to explain...

\subsection{Insight}
To exploit data locality, we must use data before it gets evicted from the cache; ideally, data is not loaded into the cache more than once.
Clearly, if the iteration space were sufficiently small, this would be a natural occurrence.
We therefore contrive small iteration spaces by partitioning the original space into smaller \emph{tiles} (Figure~\ref{fig:tiled-space}), hence the name.

\begin{figure}[h]
	\centering
	\textbf{[A nice picture of some tiles]}
	\caption{Tiles over an iteration space. Note that the tile size need not be the same in each dimension, or divide the extent of the iteration cleanly.}
	\label{fig:tiled-space}
\end{figure}

Loop tiling is also commonly known as \emph{blocking}, or perhaps less transparently \emph{strip-mine and interchange}, as tiling is typically achieved through these two transformations.\footnote{Tiling may also enable other transformations, such as loop-invariant code motion, which may not be appropriate when the extents of the iteration are too large.}

\subsection{Strip-mining}
Named after the mining practice, strip-mining involves dividing a dimension of the iteration space into strips (Figure~\ref{lst:stripmine}).\footnote{However, you cannot divide a dimension into lateral strips, only sequential ones.}
By itself, strip-mining does not change the execution order; it is a gateway to further transformations.

\begin{figure}[h]
	\textbf{A listing showing a basic loop, strip-mined.}
	\caption{A regular loop and a strip-mined loop. Note some clever things like the variable (dimension), which are the strips, introduction of min, etc.}
	\label{lst:stripmine}
\end{figure}

% TODO: consider visualising the iteration space here

\subsection{Loop interchange}
Loop interchange is based on the observation that a change in execution order does not change the correctness of a strip-mined program.
We will change the order of the loops to iterate over the blocks, then within them (Figure~\ref{lst:interchange}).

\begin{figure}[h]
	\textbf{A listing of the previous stripmined loop with interchanged loop headers.}
	\caption{The loop nest of Figure~\ref{lst:stripmine}, with the SOME LOOPS interchanged.}
	\label{lst:interchange}
\end{figure}

This is valid when each point in the iteration space does not depend on the values calculated in the same iteration.
Therefore, one must be extremely careful that no data dependencies cross boundaries between blocks; if they do, they must be permitted to cross only in one direction, and the blocks must be scheduled in that order.

\subsection{Skewing}
Data dependencies: why time-tiling is a problem

\subsection{Polyhedral model}
Maybe?

% Recall that we wanted to use the cache, because I have surely forgotten...

% Also Yask, Halide, Firedrake, Pochoir

\section{Devito}
What is Devito/Why would I use Devito?
What can Devito do?
(What am I doing?)
