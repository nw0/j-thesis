\documentclass[thesis.tex]{subfile}

\chapter{Introduction}
% Limit myself to 2 paras here?

In many engineering domains, finite-difference methods are used to solve differential equations, where an exact solution may be computationally infeasible.
These approximations obviate the need for equations to be solved by hand, saving considerable labour.
Optimising the computations is therefore a natural extension of this task.

% TODO: look up Devito's raison d'etre
For scientists using differential equations in novel ways, tools which can compute partial differential equations directly are helpful, as it enables them to refine their models quickly.
Such tools are less crucial to those employing fluid-dynamical models, which are relatively well-understood and optimised. % TODO: rewrite, cite


\section{Motivation}
% Talk about the big idea in background
% Here: why/who optimise?

It is inefficient and arguably undesirable for scientists principally concerned with modelling to need to understand the structures which speed up their computation. % "understand": think about, etc
As a means of abstraction, compilers are used to transform differential equations into stencils, then code to evaluate the equations.
This reduces the effort required to comprehend the calculations, aiding maintenance and even reproducibility; generating and optimising code by hand is rarely feasible or efficient~\cite{olgaard10}.

Decoupling the understanding of optimisations means that domain specialists outside of software performance and compiler technology need not keep up with computer architecture.
New instructions may be introduced (such as vectorisation); not for nothing are compilers able to generate code tailored to the architecture for which a program is compiled.

\section{Devito}
% Optimise how?

% Vertical integration: to simplify this workflow, Devito starts from PDEs
% Devito because it allows scientists to change their models easily
% [Remark: fluid dynamics people never change PDEs, already handcoded & fast]
% Context: other stencil DSLs do time tiling starting from stencils
% Allows us to generate meaningful problem instances (don't write this here)

Devito, a tool for efficient application of finite-difference methods, is able to generate computations directly from differential equations, achieving a notion of `vertical integration' within the modelling ecosystem.
This is extremely helpful if one is experimenting with models, or continually modelling new problems, as one can change the equations used and rapidly generate and execute the relevant computation.

% TODO: Check definition of stencil
In modern usage, such computations may be transformed into \emph{stencils}.
There are many compilers optimising stencil kernels with \emph{time-tiling}, an optimisation which has not been implemented in Devito yet.
% TODO: why it hasn't been solved for Devito
It has been previously demonstrated that time-tiling has the potential to reduce the run time of `some Devito stencil loops by up to 27.5\%' \cite{dylan} when applying time-tiling against a baseline of tiling in all other dimensions, which the tool was already able to perform.
In this case, another tool (CLooG) was used to perform the tiling.
We extend tiling to the time dimension natively in Devito to realise this speedup integrated with its other optimisations, and evaluate the claim.


\section{Objectives and (intended) contributions}
The principal objective of this project was to implement tiling over the time dimension in Devito and evaluate its performance.

In summary, our contributions are:

% TODO: We want a lot more detail in this list
\begin{itemize}
	\item A survey of the background required for time-tiling, and related work. A consideration of the implications of and motivation for time-tiling.
	(Chapter~\ref{ch:background})
	\item An implementation of skewing and time-tiling in Devito. An analysis of its legality and the necessary checks to guarantee this, and a set of relevant test cases.
	(Chapter~\ref{ch:implementation})
	\item An evaluation of correctness and performance of the time-tiling transformation against the original claim and other orderings, and any actions end-users may need to take to realise performance gains.
	(Chapter~\ref{ch:evaluation})
	\item A discussion on further implementation work and integration with other optimisations.
	(to be determined)
\end{itemize}
