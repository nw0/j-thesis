\documentclass[thesis.tex]{subfile}

\chapter{Evaluation}
\label{ch:evaluation}

This chapter is devoted to the evaluation of performance of time-tiling as implemented in Chapter~\ref{ch:implementation}.
We discuss the functionality of the time-tiling implementation and its technical requirements (Section~\ref{sec:time-tiling-reqs}), and the test methodology (Section~\ref{sec:test-method}).
Section~\ref{sec:perf-laplace} examines results under the Laplacian operator.
Finally, we discuss limitations of testing (Section~\ref{sec:perf-limitations}), and consider further work in evaluation (Section~\ref{sec:further-eval}).


\section{Pre-requisites for time-tiling}
\label{sec:time-tiling-reqs}

% Nothing but Devito
% Time buffering and save=n>2*time tile size (a nice diagram!)
% Further prereqs to use the AT


\section{Testing methodology}
\label{sec:test-method}

% Tern, openmp, icc, cat envs
% Usage of AT -- parameters (skewing factor, time and space tile sizes)
% Discuss roofline model


\section{Performance under the Laplace operator}
\label{sec:perf-laplace}


\section{Limitations of performance evaluation}
\label{sec:perf-limitations}


\section{Further evaluation}
\label{sec:further-eval}


\paragraph{BREAK}
Everything that follows here is old.

Previous evaluation suggested that Devito gains considerable speedup from time-tiling.
This chapter focusses on evaluating this statement with general time-tiling implemented in Devito.

\paragraph{Objective}
We want to evaluate the performance of the time-tiling transformation against that of tiling only in the other dimensions.


\section{Functional correctness}
This will be done primarily using test cases, some of which have already been written to test the features already implemented.
Where possible, existing test cases are reused, when the output is expected not to vary.

Additionally, while Devito is targeted at seismic imaging, it can also be used to apply finite-difference methods to general differential equations.
We are therefore able to perform testing against many natural examples, rather than being dependent on hand-crafted stencils.


\section{Initial performance evaluation}
The initial evaluation was completed using a machine in the Department of Computing Student Lab (\texttt{graphic12}).
We used an iteration space of dimensions \(384^3\) under a Laplacian operator, larger than the cache size to ensure that the tiling functionality was utilised.
A 30.7\% decrease in runtime was recorded with the Intel C++ Compiler (31.5\% under GCC) over 30 runs.

We will perform evaluation against a variety of stencils, including the acoustic wave equation (AWE) stencil used in~\cite{dylan}.
In all cases, consideration will be necessary to deal with time-dimension buffering.
This evaluation would comprise both timing and memory usage, as well as examining the loop structures of the generated code.

For a more in depth analysis, we would consider memory analyses measuring cache misses and memory traffic, to ensure that improvements are indeed gleaned from time-tiling rather than other factors.


\section{Extended evaluation}
As mentioned above, Devito gives rise to many natural test cases.
Implementation and time permitting, we could evaluate the benefit of time-tiling against a plethora of problems, such as inversion, or other operator orderings.
