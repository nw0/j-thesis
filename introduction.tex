\documentclass[thesis.tex]{subfile}

\chapter{Introduction}
% Limit myself to 2 paras here?

\section{Motivation}

In many engineering domains, finite-difference methods are used to find approximate solutions to differential equations.
For scientists using differential equations in novel ways, tools which can transform partial differential equations to optimised computations and evaluate them are useful; this enables them to refine their models quickly.

It is inefficient and undesirable for specialists principally concerned with modelling to optimise their computations by hand.
As a means of abstraction, compilers can be used to automatically transform differential equations into stencils, then code to evaluate the equations.
A tool automatically performing this transformation saves considerable labour and scope for errors, and reduces the effort required to comprehend the calculations.
These aid maintenance and even reproducibility; generating and optimising code by hand is rarely feasible or efficient~\cite{olgaard10}.

Decoupling optimisation from specific models means that domain specialists can take advantage of optimisations seamlessly without intervention, allowing them to focus their attention and expertise on their own domain.
They can easily take advantages of the latest advances in computer architecture and new instructions such as vectorisation, which give rise to new optimisations in finite-difference methods.

\section{Devito}
% [Remark: fluid dynamics people never change PDEs, already handcoded & fast]
% Context: other stencil DSLs do time tiling starting from stencils

In modern usage, the partial differential equations are transformed into \emph{stencils}, which define the computation, then code to execute it.
One of the challenges faced by domain specialists generating new models is the time necessary to perform this translation.
Thus the desire for a compiler to save time and effort arises.

Devito~\cite{devito}, a tool for efficient application of finite-difference methods, is able to generate computations directly from differential equations, achieving a notion of `vertical integration' within the modelling ecosystem.
There are several compilers, a number of which we review in Chapter~\ref{ch:background}, which perform optimisation on stencil kernels, but none which also transform differential equations into stencils.
Being able to automate this transformation is extremely helpful if one is experimenting with models, or continually modelling new problems, as one can change the equations used and rapidly generate and execute the relevant computation.

Some of these stencil compilers are able to apply the \emph{time-tiling} optimisation, which has not been implemented in Devito yet.
It has been previously demonstrated that time-tiling has the potential to reduce the run time of `some Devito stencil loops by up to 27.5\%'~\cite{dylan} when performing time-tiling against the existing spatial tiling which the tool is able to perform.
In this case another tool, CLooG~\cite{cloog-isl}, was used to add time-tiling to Devito-generated code.

\clar{Too much detail?}

While the literature surrounding the polyhedral model, from which time-tiling derives, dates from the last century, it was viewed as complicated and time-consuming to use in optimising compilers as recently as 2004 when Bastoul proposed extensions~\cite{cloog-isl} to the original Quiller\'e et al.~algorithm to eliminate redundant (generated) code~\cite{quillere}.
Time-tiling was been studied in the form this work considers since Griebl in 2004~\cite{griebl-tt}, using \emph{space-time mapping}, analogous to the skewed tiling which we discuss in this work, but expressed as affine conditions specifying hyperplane partitions of the problem domain.

Time-tiling has since been studied extensively, and long implemented in polyhedral compilers such as CLooG, providing a strong theoretical foundation.
However, the actual application of the technique beyond theoretical study has, until this work, been limited to the OPS (Oxford Parallel Structured software) project~\cite{ops-main}, whose investigations into time-tiling commenced just 4 years ago, in 2014~\cite{ops-tt-cacheblocking}.
Even so, OPS is a source-to-source compiler, operating from the stencil kernel level down to code generation; Devito includes all these plus layers of abstraction up to the differential equations themselves.

Therefore, we extend tiling to the time dimension natively in Devito to realise this performance gain and evaluate its efficacy.
This gives the performance gains of time-tiling to Devito without the need to configure another tool to perform further optimisations.


\section{Contributions}
The principal objective of this project was to implement tiling over the time dimension in Devito and evaluate its performance against the existing optimisation, that of tiling restricted to the spatial dimensions.

In summary, our contributions are:

\wip{Results change; check this late}
\begin{itemize}
	\item An implementation of time-tiling in Devito, including a simplification of the generated loop structure, with accompanying test cases and auto-tuner enhancements.
	An analysis of its legality and the necessary conditions to guarantee this.

	\item An evaluation of the correctness and performance of the time-tiling transformation and any actions end-users may need to take to realise performance gains.

	\item Demonstration of a runtime decrease of up to 25\% compared to Devito's existing optimisations under real-world applications, in particular the stencils from the family of acoustic wave equation operators, widely used in Devito's target domain, seismic imaging.
	A model to understand and maximise the runtime reductions arising from time-tiling these stencils.

	\item A novel estimator for \emph{arithmetic intensity under time-tiling} that is consistent with the existing estimator for arithmetic intensity under spatial tiling in Devito.
	Bounds for both estimators and an evaluation of their utility.

	\item A discussion on further implementation work, analysing their importance and consistency with the transformation we have implemented.
\end{itemize}


\section{Report structure}
This report is divided into 6 chapters, describing the above contributions and the context from which they arise.

\wip{Rewriting this: noted Paul on possible further gains}
\begin{description}
	\item[Chapter~\ref{ch:background}]
	A review of the fundamentals needed for time-tiling in general, including its motivation and consequences.
	A survey of work related to stencil compilers, approximate solvers of partial differential equations, in particular the abstractions that they make available to the user.

	\item[Chapter~\ref{ch:devito}]
	Details of the existing Devito compilation process which will be affecting by time-tiling, and its existing implementation of spatial tiling.

	\item[Chapter~\ref{ch:implementation}]
	The implemented time-tiling transformation.
	Explanations of the design choices made, and modifications to the auto-tuner to ensure time-tiling is used effectively.
	Proposals for further improvement arising from time-tiling.

	\item[Chapter~\ref{ch:evaluation}]
	An analysis of the effectiveness of the implemented transformation.
	Evidence of performance gains, and the conditions under which they are most significant.

	\item[Chapter~\ref{ch:conclusion}]
	The conclusion of the work, a discussion of future work to be done, and the streamlining of this optimisation process.
\end{description}
