\documentclass[thesis.tex]{subfile}

\chapter{Introduction}
% Limit myself to 2 paras here?

In many engineering domains, finite-difference methods are used to find approximate solutions to differential equations.
For scientists using differential equations in novel ways, tools which can compute partial differential equations directly are helpful, as it enables them to refine their models quickly.

As computer architecture evolves and new instructions, such as vectorisation, are introduced, new optimisations are possible in finite-difference methods.


\section{Motivation}
It is inefficient and undesirable for specialists principally concerned with modelling to need to understand the structures which speed up their computation.
As a means of abstraction, compilers are used to automatically transform differential equations into stencils, then code to evaluate the equations.
This reduces the effort required to comprehend the calculations, aiding maintenance and even reproducibility; generating and optimising code by hand is rarely feasible or efficient~\cite{olgaard10}.

Decoupling optimisation from specific models means that domain specialists outside of software performance and compiler technology need not deal with architectural changes and optimise manually.

\section{Devito}
% [Remark: fluid dynamics people never change PDEs, already handcoded & fast]
% Context: other stencil DSLs do time tiling starting from stencils

One of the challenges faced by those using differential equations to generate new models is the time necessary to differential equations into stencils and then code for computation.

Devito~\cite{devito}, a tool for efficient application of finite-difference methods, is able to generate computations directly from differential equations, achieving a notion of `vertical integration' within the modelling ecosystem.
This is extremely helpful if one is experimenting with models, or continually modelling new problems, as one can change the equations used and rapidly generate and execute the relevant computation.

% TODO: Check definition of stencil
In modern usage, such computations may be transformed into \emph{stencils}.
There are many compilers optimising stencil kernels with \emph{time-tiling}, an optimisation which has not been implemented in Devito yet.
% TODO: why it hasn't been solved for Devito
It has been previously demonstrated that time-tiling has the potential to reduce the run time of `some Devito stencil loops by up to 27.5\%' \cite{dylan} when performing time-tiling against the existing spatial tiling which the tool is able to perform.
In this case, another tool (CLooG) was used to add time-tiling to the Devito code.

Therefore, we extend tiling to the time dimension natively in Devito to realise this performance gain and evaluate its efficacy.
This removes the need to configure another tool each time Devito is run.


\section{Contributions}
The principal objective of this project was to implement tiling over the time dimension in Devito and evaluate its performance against the existing optimisation, that of tiling restricted to the spatial dimensions.

In summary, our contributions are:

% TODO: We want a lot more detail in this list
\begin{itemize}
	\item An implementation of time-tiling in Devito, including a simplification of the generated loop structure.
	An analysis of its legality and the necessary checks to guarantee this.
	We also include test cases to supplement the existing test suite for the skewing and tiling transformations which we apply.

	\item An evaluation of correctness and performance of the time-tiling transformation and any actions end-users may need to take to realise performance gains.

	\item Demonstration of a runtime decrease of at least 25\% due to the application of time-tiling in Devito.
	An evaluation on the conditions conducive to maximising the performance increase.

	\item A discussion on further implementation work and integration with other optimisations.
\end{itemize}


\section{Report structure}
This report is divided into 6 chapters, describing the above contributions and the context from which they arise.

\begin{description}
	\item[Chapter~\ref{ch:background}]
	A review of the fundamentals needed for time-tiling in general, including its motivation and consequences.
	A survey of work related to stencil compilers, approximate solvers of partial differential equations, in particular the abstractions that they make available to the user.

	\item[Chapter~\ref{ch:devito}]
	Details of the existing Devito compilation process which will be affecting by time-tiling, and its existing implementation of spatial tiling.

	\item[Chapter~\ref{ch:implementation}]
	The implemented time-tiling transformation.
	Explanations of the design choices made, and modifications to the auto-tuner to ensure time-tiling is used effectively.
	Proposals for further improvement arising from time-tiling.

	\item[Chapter~\ref{ch:evaluation}]
	An analysis of the effectiveness of the implemented transformation.
	Evidence of performance gains, and the conditions under which they are most significant.

	\item[Chapter~\ref{ch:conclusion}]
	The conclusion of the work, and consider possible future improvements to time-tiling in Devito, and the streamlining of this optimisation process.
\end{description}
