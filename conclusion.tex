\documentclass[thesis.tex]{subfile}

\chapter{Conclusion}
\label{ch:conclusion}


\section{Context and review}

This project drew together aspects of compiler transformations, differential equations and computation, computer architecture, and software engineering.
The different perspectives of the fields provided immense insight into the complexity of numerically approximating partial differential equations---a far cry from anything Taylor may have imagined in the 18th century!
These insights culminated in a transformation yielding significant performance gains in Devito's target domain: seismic imaging.
Though time-tiling is a well-established optimisation, this implementation is a necessary and important step for Devito.

\wip{Rewrite; move?}

During our evaluation in Chapter~\ref{ch:evaluation}, we used this abstraction to effortlessly generate entire families of stencils by varying the space order; producing these stencils from the equations is decidedly non-trivial, and would have otherwise been extremely time-consuming.
As a consequence, this project includes the broadest survey of stencils under time-tiling to date.

While the concept of time-tiling is superficially straightforward, its implementation was decidedly not so: unlike interfacing with a `back-end' polyhedral compiler, such as CLooG or Pochoir (Section~\ref{sec:pochoir}) with all its attendant integration problems, implementation within Devito meant understanding the numerous stages of internal representation used to transform a differential equation into code.
Without these layers, Devito would not be able to expose APIs at different levels for more intricate applications; making a transformation that would work with these was a major hurdle.

The performance evaluation was enlightening and challenging in its own ways.
It is easy to be convinced of the benefits of time-tiling from the literature, and another experience altogether to witness real decreases in runtime while watching a wall clock;
mustering the scepticism required to confirm its validity required much deeper reference to computer architecture and computation than implementing the transformation had needed.


\section{Contributions}
A summary of the contributions of this project.

\wip{Results change, proofread this late}
\begin{itemize}
	\item The headline contribution of the project; documentation and implementation of the time-tiling transformation for perfect loop nests in Devito, with accompanying test cases and auto-tuner enhancements.
	This is a necessary and significant step toward implementing time-tiling for Devito in its full generality, including tiling of imperfect loop nests.

	\item We outlined the remaining work in implementation, including analyses of their importance and difficulty, and their consistency with our implemented transformation.

	\item We presented and critiqued our testing methodology and models, which we used for the performance evaluation of our implemented time-tiling transformation.
	In particular, we devoted attention to numerical verification, the set-up of the test environment, and the modelling of performance.

	\item We demonstrated a runtime decrease of 25\% with the transformation compared to Devito's existing optimisations under real-world applications, particularly the stencils of varying space orders arising from the acoustic wave equation, of great importance to Devito's target domain, seismic imaging.
	In so doing, we validated previous work which demonstrated a runtime reduction of up to 27.5\%~\cite{dylan}.

	\item Further, we developed a model estimating runtime reductions based on the space order of a stencil generated from the family of acoustic wave equation operators.

	\item We proposed an estimator for \emph{arithmetic intensity under time-tiling} that is consistent with Devito's reported arithmetic intensity under spatial tiling.
	Just as importantly, we established bounds for both and showed that the estimators are plausibly close to the true arithmetic intensity of a stencil computation, and that they are consistent with the widely-cited roofline model.

	\item We developed a hypothesis on how the skewing factor affects the runtime of a stencil, which holds in our test architecture, and in particular our experimentation.
\end{itemize}

\section{Future work}
\label{sec:future-work}

Apart from work in evaluation and extensions to time-tiling discussed below, we also like to highlight Section~\ref{sec:impl-minor} (`Minor extensions to time-tiling').

\wip{eval}

\subsection{Tiling imperfectly-nested loops}
In this work, we implemented the time-tiling transformation for perfect loop nests in Devito.
However, Devito is also used to solve problems that do not naturally generate perfect loop nests.
An example would be source and receiver loops, which may be used with the acoustic wave equation to model the propagation of waves to study unknown media.
We briefly discussed this additional transformation in Section~\ref{sec:impl-imperfect}.

\wip{lst:imperfect-loop}

Figure TODO gives an example of generated source and receiver loops in Devito.
Our tiling implementation is not able to time-tile this loop, although it may spatially tile the loop nest.

\subsubsection{Strategy for tiling imperfectly-nested loops}
In order to tile an imperfectly-nested loop, it must be transformed into a perfect loop nest.
To do this, transformations such as loop fission (previously discussed), loop fusion, and code sinking are used.
Ahmed et al.~generalise these transformations as conditionals in the body of the innermost loop when iterating over the \emph{product space}~\cite{ahmed-imperfect}, which is generated by the original variables of the imperfectly-nested loop; this generates a perfect loop nest of higher dimension, or a deeper loop nest.
Lim illustrates this particularly well using a slightly different strategy, affine partitioning~\cite{lim-affine-part}.

Again, the concepts for transforming imperfectly-nested loops to perfectly-nested loops are intuitive, but potentially tedious to generalise.
Ahmed explicitly provides an algorithm for doing so and analysis of the generated structure~\cite{ahmed-synth}; neither is remotely trivial.
In particular, evaluation should be performed to ascertain any runtime improvement between time-tiled stencils needing source and receiver loops and the same loops under spatial tiling.

\subsubsection{Consistency with time-tiling}
Manipulating a loop nest involves additional transformations.
It is clear from the literature that these are semantically compatible with tiling.
For skewing, consider that no change has been made to the execution order; loop index offsets are balanced by an equal and opposite change in loop variables.
Therefore, the necessary code transformations are not affected by skewing.


\subsection{Advanced manipulation of symbolic expressions}
\label{sec:future-aggressive}

The Devito symbolic engine (DSE) has additional modes which we have not discussed in detail in this report.
These enable it to perform more advanced transformations on the symbolic expressions that make up a differential equation.
In particular, they search for sub-expressions that should be assigned to temporaries, to avoid needless re-computation.

Our implementation separates skewing into a separate DSE mode, so that it will not occur alongside these advanced transformations.

\subsubsection{Consistency with time-tiling}
First, observe that these transformations were compatible with spatial tiling which was previously implemented using the remainder loop strategy.
Since it is legal to permute the spatial tiles within a single time iteration, our bounding strategy is guaranteed to be valid under spatial tiling.

For time-tiling, consider that our implementation does not parallelise the time dimension.\footnote{This is a consequence of our skewing factor being \emph{equal} to the largest spatial dependence distance. Increasing this by one would allow interchange of the incremental time loop with any increment spatial loop, however at least one incremental loop must not be parallelised.}
Thus the extraction will be valid, as all extracted code will be placed within the incremental time loop.

\subsubsection{Possible pitfalls in implementation}
This transformation extracts sub-expressions from \emph{SymPy} expressions, potentially changing the order of floating-point operations.
An obvious solution would be to perform skewing after this transformation, as we established in this work that skewing does not change the execution order.
A caveat is that skewing affects the behaviour of common sub-expression elimination (Section~\ref{sec:impl-cse}), which \emph{does} result in an execution order change in Devito.

This should be overcome with using a test case with suitable bounds; as Devito does not guarantee an order of operations on floats (due to optimisation), differences in values due to re-ordering are acceptable in principle.

A final remark is that skewing manifests in two parts: loop indices are skewed in the DSE, but the loop variable is not skewed, as the loop structure has not been generated yet.
Instead, the skewing factor is maintained as an attribute of the stencil until it is transformed into a loop.
This fact may explain how skewing and the other optimisations change the execution order, and it might even be possible to surmount this change.
